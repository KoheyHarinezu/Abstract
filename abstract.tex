% Windowsでコンパイルすること

\documentclass[11pt,dvipdfmx]{jarticle}

\usepackage{tikz}
\usetikzlibrary{intersections, calc, arrows}
\usepackage{multicol}
\usepackage{titlesec}
\titleformat*{\section}{\large\bfseries}

%%%%%%%%%%%%%%%%%%%%%%%%%%%%%%%%%%%%%%%%%%%%%%%%%%%%%%%%%
\arrayrulewidth = 1pt
\newcommand\setTBstruts{\def\T{\rule{0pt}{3ex}}
                        \def\B{\rule[-2.3ex]{0pt}{0pt}}
                        \def\A{\rule{0pt}{5ex}}
                        \def\C{\rule[-5ex]{0pt}{0pt}}}

\setlength{\textheight}{255mm}
\setlength{\oddsidemargin}{-10mm}
\setlength{\textwidth}{180mm}
\setlength{\topmargin}{-20mm}
\setlength{\columnsep}{1cm}
\setTBstruts
\title{\vspace{-2mm}\fontsize{14pt}{0pt}\selectfont 論文概要\vspace{-4.1mm}}
\author{
	\begin{tikzpicture}
		\draw[line width=1.4pt] (0,0)--(16,0)--(16,3.15)--(0,3.15)--cycle;
		\draw[line width=1.4pt] (0,1.05)--(16,1.05);
		\draw[line width=1.4pt] (0,2.10)--(16,2.10);
		\draw[line width=1.4pt] (2.49,0)--(2.49,3.15);
		\draw[line width=1.4pt] (8,1.05)--(8,3.15);
		\draw[line width=1.4pt] (8+2.49,1.05)--(8+2.49,3.15);
		\draw(1.245,2.10+0.525) circle(0) node{学科};
		\draw(1.245,1.05+0.525) circle(0) node{学生番号};
		\draw(1.245,0.525) circle(0) node{題目};
		\draw(1.245+8,2.10+0.525) circle(0) node{指導教員};
		\draw(1.245+8,1.05+0.525) circle(0) node{氏名};
		\draw(1.245,2.10+0.525) circle(0) node{学科};
		\draw(1.245,1.05+0.525) circle(0) node{学生番号};
		\draw(5.745-0.5,2.10+0.525) circle(0) node{
		% 学科名を書く
			12345678
		};
		\draw(5.745-0.5,1.05+0.525) circle(0) node{
		% 学籍番号を書く(大文字)
			12345678
		};
		\draw(5.745-0.5+8,2.10+0.525) circle(0) node{
		% 氏名を書く(全角スペースを入れる)
			A B C D
		};
		\draw(5.745-0.5+8,1.05+0.525) circle(0) node{
		% 指導教員名を書く(全角スペースを入れる)
			E F G H
		};
		\draw(1.245+8,0.525) circle(0) node{
		% 題目を書く
			ABCDEFG
		};
	\end{tikzpicture}
}
\date{}
%%%%%%%%%%%%%%%%%%%%%%%%%%%%%%%%%%%%%%%%%%%%%%%%%%%%%%%%%


\begin{document}
%1段組みにしたいならコメントアウト
\twocolumn
\maketitle

%%%%%%%%%%%%%%%%%%%%%%%%%%%%%%%%%%%%%%%%%%%%%%%%%%%%%%%%%
\vspace{-10mm}
%%%%%%%%%%%%%%%%%%%%%%%%%%%%%%%%%%%%%%%%%%%%%%%%%%%%%%%%%
%項目数を増やしたい場合はここを増やす
\noindent
\hspace{-4.5mm}
\fontsize{12.5pt}{0pt}\selectfont
\special{pdf:literal direct q 0.3 w 2 Tr}
\special{pdf:bcolor [0 0 0]}
%書き換える スペースを入れる
1 はじめに
\special{pdf: literal direct Q}
\special{pdf:ecolor}\\
%%%%%%%%%%%%%%%%%%%%%%%%%%%%%%%%%%%%%%%%%%%%%%%%%%%%%%%%%
\fontsize{11pt}{18pt}\selectfont
\indent
%本文はここから
1234567890
1234567890
1234567890
\mbox{}\\\mbox{}\\
%%%%%%%%%%%%%%%%%%%%%%%%%%%%%%%%%%%%%%%%%%%%%%%%%%%%%%%%%
\noindent
\hspace{-3.5mm}
\fontsize{12.5pt}{0pt}\selectfont
\special{pdf:literal direct q 0.3 w 2 Tr}
\special{pdf:bcolor [0 0 0]}
%書き換える
2
\special{pdf: literal direct Q}
\special{pdf:ecolor}\\
%%%%%%%%%%%%%%%%%%%%%%%%%%%%%%%%%%%%%%%%%%%%%%%%%%%%%%%%%
\fontsize{11pt}{18pt}\selectfont
\indent
%本文はここから
1234567890
1234567890
1234567890
\mbox{}\\\mbox{}\\
%%%%%%%%%%%%%%%%%%%%%%%%%%%%%%%%%%%%%%%%%%%%%%%%%%%%%%%%%
\noindent
\hspace{-3.5mm}
\fontsize{12.5pt}{0pt}\selectfont
\special{pdf:literal direct q 0.3 w 2 Tr}
\special{pdf:bcolor [0 0 0]}
%書き換える
3
\special{pdf: literal direct Q}
\special{pdf:ecolor}\\
%%%%%%%%%%%%%%%%%%%%%%%%%%%%%%%%%%%%%%%%%%%%%%%%%%%%%%%%%
\fontsize{11pt}{18pt}\selectfont
\indent
%本文はここから
1234567890
1234567890
1234567890
\mbox{}\\\mbox{}\\
%%%%%%%%%%%%%%%%%%%%%%%%%%%%%%%%%%%%%%%%%%%%%%%%%%%%%%%%%

\end{document}
